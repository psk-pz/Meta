\documentclass[11pt,a4paper]{article}
\usepackage{polski}
\usepackage[utf8]{inputenc}
\usepackage{hyperref}

\begin{document}
\begin{titlepage}
    \centering
    {\scshape\huge Politechnika Świętokrzyska \par}
    \vspace{0.5cm}
    {\Large\itshape Daniel Iwaniec, Artur Kałuża, Karol Gos, Karol Gołuch\par}
    \vspace{2cm}
    {\scshape\LARGE Projekt zespołowy\par}
    \vspace{0.5cm}
    {\bfseries System zgłoszeń dla serwisu sprzętu elektronicznego\par}

    \vfil

    \raggedright
    \section{Wstęp}
    Projekt zakłada wytworzenie oprogramowania w architekturze \textbf{Representational State Transfer}, w skład którego wchodzi serwer zaimplementowany w technologii \textit{PHP} oraz klient zaimplementowany w~technologiach \textit{HTML/CSS/JavaScript}.
    \par \bigskip Implikuje to również zaprojektowanie i wykonanie odpowiedniego \textbf{Domain~Application~Protocol} potrzebnego do komunikacji pomiędzy wymienionymi warstwami.
    \par \bigskip Kod źródłowy projektu jest udostępniony na licencji \textit{MIT}:
    \par \underline{\href{https://github.com/psk-pz}{github.com/psk-pz}}.
\end{titlepage}

\section{Aktorzy}
\begin{enumerate}
\item{\textbf{Użytkownik} - klient serwisu.}
\item{\textbf{Serwisant} - pracownik serwisu.}
\item{\textbf{Administrator} - administrator systemu.}
\end{enumerate}

\section{Encje}
\begin{enumerate}
\item{\textbf{Zgłoszenie} - zgłoszenie utworzone w systemie przez \textit{użytkownika}. Po utworzeniu przechodzi przez dozwolone stany, aż do jego zamknięcia.}
\item{\textbf{Notyfikacja} - notyfikacje wysyłane \textit{serwisantom} oraz \textit{użytkownikom} w momencie wystąpienia określonych zdarzeń.}
\end{enumerate}

\section{Stany zgłoszenia}
\begin{enumerate}
\item{\textbf{Otwarte} - nowo utworzone zgłoszenie.}
\item{\textbf{Odrzucone} - odrzucone przez \textit{serwisanta} lub \textit{użytkownika}.}
\item{\textbf{Ponownie otwarte} - ponownie otwarte po uprzednim odrzuceniu.}
\item{\textbf{Przyjęte} - przyjęte prez \textit{serwisanta} do wyceny.}
\item{\textbf{Oczekuje na akceptację} - oczekuje na akceptację kosztów.}
\item{\textbf{Zaakceptowane} - zaakceptowane prez \textit{użytkownika}.}
\item{\textbf{W trakcie pracy} - w trakcie naprawy sprzętu przez \textit{serwisanta}.}
\item{\textbf{Nie można naprawić} - odesłany nienaprawiony sprzęt.}
\item{\textbf{Wysłane} - odesłany naprawiony sprzęt.}
\item{\textbf{Zakończone} - zgłoszenie zostaje zarchiwizowane.}
\end{enumerate}

\vfill

\section{Scenariusz odrzucenia zgłoszenia}
\begin{enumerate}
\item{\textit{Użytkownik} odwiedza stronę z formularzem do utworzenia zgłoszenia.}
\item{\textit{Użytkownik} wysyła dane do utworzenia zgłoszenia.}
\item{Jeżeli wystąpiły błędy:}
\begin{enumerate}
\item{Błędy są wyświetlane na ekranie.}
\item{\textit{Użytkownik} poprawia dane oraz powraca do kroku 2.}
\end{enumerate}
\item{Wyświetlany jest komunikat o sukcesie oraz identyfikator zgłoszenia.}
\item{Tworzona jest notyfikacja o utworzeniu nowego zgłoszenia w systemie widoczna dla \textit{serwisantów}.}
\end{enumerate}

\section{Scenariusz odczytania stanu zgłoszenia}
\begin{enumerate}
\item{\textit{Użytkownik} odwiedza stronę z formularzem wymagającym identyfikatora zgłoszenia.}
\item{\textit{Użytkownik} wysyła identyfikator zgłoszenia.}
\item{Jeżeli zgłoszenie nie istnieje:}
\begin{enumerate}
\item{Błąd jest wyświetlane na ekranie.}
\item{\textit{Użytkownik} powraca do kroku 1.}
\end{enumerate}
\item{Wyświetlany jest aktualny stan zgłoszenia oraz jego historia.}
\item{Wyświetlane są również notyfikacje dotyczące zgłoszenia.}
\end{enumerate}

\section{Scenariusz odrzucenia wyceny}
\begin{enumerate}
\item{\textit{Serwisant} otrzymuje notfikację o nowym zgłoszeniu.}
\item{\textit{Serwisant} przyjmuje zgłoszenie do wyceny.}
\item{\textit{Serwisant} wycenia koszta naprawy oraz przekazuje do akceptacji przez \textit{użytkownika}.}
\item{\textit{Użytkownik} otrzymuje notyfikację o wycenia do zaakceptowania.}
\item{\textit{Użytkownik} odrzuca koszta wyceny.}
\item{\textit{Serwisant} zamyka zgłoszenie.}
\end{enumerate}

\section{Scenariusz wykonania naprawy}
\begin{enumerate}
\item{\textit{Serwisant} otrzymuje notfikację o nowym zgłoszeniu.}
\item{\textit{Serwisant} przyjmuje zgłoszenie do wyceny.}
\item{\textit{Serwisant} wycenia koszta naprawy oraz przekazuje do akceptacji przez \textit{użytkownika}.}
\item{\textit{Użytkownik} otrzymuje notyfikację o wycenia do zaakceptowania.}
\item{\textit{Użytkownik} akceptuje koszta wyceny.}
\item{...}
\item{\textit{Użytkownik} zamyka zgłoszenie.}
\end{enumerate}

\end{document}
